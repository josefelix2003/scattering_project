\documentclass{article}
\usepackage[utf8]{inputenc}
\usepackage{amsmath}
\usepackage{amssymb}
\usepackage[french]{babel}
\usepackage{xcolor}
\usepackage{tcolorbox}
\usepackage{cancel} % pour barrer


\begin{document}

\section*{Introduction :}
\bigskip
\textit{"On désire construire un code de calcul capable de simuler la diffraction d’une onde EM plane
par un réseau de diffraction plantaire fait de strips à base de graphène."
}
\section*{Équations de Maxwell}

On se place dans un milieu LHI (Linéaire, Homogène, Isotrope) :

\[
\left\{
\begin{aligned}
\vec{\nabla} \cdot \vec{D} &= \rho \\
\vec{\nabla} \cdot \vec{B} &= 0 \\
\vec{\nabla} \wedge \vec{E} &= -\frac{\partial \vec{B}}{\partial t} \\
\vec{\nabla} \wedge \vec{H} &= \frac{\partial \vec{D}}{\partial t}
\end{aligned}
\right.
\]

\texf{Avec les relations :}

\[
\left\{
\begin{aligned}
\vec{D} &= \varepsilon \vec{E} = \varepsilon_0 \varepsilon_r \vec{E} \\
\vec{B} &= \mu \vec{H} = \mu_0 \mu_r \vec{H}
\end{aligned}
\right.
\]

En remplaçant \(\vec{B}\) et \(\vec{D}\) dans les équations de Maxwell :

\[
\left\{
\begin{aligned}
\vec{\nabla} \wedge \vec{E} &= - \mu_0 \mu_r \frac{\partial \vec{H}}{\partial t} \\
\vec{\nabla} \wedge \vec{H} &= \varepsilon_0 \varepsilon_r \frac{\partial \vec{E}}{\partial t}
\end{aligned}
\right.
\]

\section*{Hypothèse harmonique}

On sépare les variables temporelles et spatiales. On fait l’hypothèse que les champs oscillent de manière sinusoïdale à une pulsation unique \(\omega\) et ceux depuis la nuit des temps. Par ailleurs, le fait de rendre la partie spatiale indépendante de la pulsation (et donc de la période) permettra, dans la suite de ce projet, d'utiliser la théorie de Fourier.

\[
\vec{\mathcal{E}}(\vec{r}, t) = \vec{E}(\vec{r}) e^{-i \omega t}, \quad
\vec{\mathcal{D}}(\vec{r}, t) = \vec{D}(\vec{r}) e^{-i \omega t}, \quad
\vec{\mathcal{B}}(\vec{r}, t) = \vec{B}(\vec{r}) e^{-i \omega t}, \quad
\vec{\mathcal{H}}(\vec{r}, t) = \vec{H}(\vec{r}) e^{-i \omega t}
\]

\[
\frac{\partial \vec{\mathcal{E}}(\vec{r}, t)}{\partial t} = -i \omega \vec{\mathcal{E}}(\vec{r}, t), \qquad
\frac{\partial \vec{\mathcal{H}}(\vec{r}, t)}{\partial t} = -i \omega \vec{\mathcal{H}}(\vec{r}, t)
\]

\[
\left\{
\begin{aligned}
\vec{\nabla} \wedge \vec{E} &= i \omega \mu_0 \mu_r \vec{H} \\
\vec{\nabla} \wedge \vec{H} &= - i \omega \varepsilon_0 \varepsilon_r \vec{E}
\end{aligned}
\right.
\]
\section*{Incidence normale : }
La forme du réseau impose la base cartésienne pour traiter le problème.

Le réseau est tout d'abord invariant suivant la direction \(y\). Les champs ne dépendent donc pas de \(y\). Par ailleurs, le théorème de Noether stipule qu'une invariance en translation est à l'origine de la conservation de la quantité de mouvement. Nous avons un problème périodique, c'est donc le vecteur d'onde \(\vec{k}\) qui est conservé. Ce vecteur est donc de norme constante. La composante \(k_y\) est nulle car on est en incidence normale. Une composante constante et nulle en un point est nulle de partout.

Nous pouvons donc calculer les rotationnels de \(\vec{E}\) et \(\vec{H}\) en ignorant les dérivées par rapport à \(y\) :

\[
\left\{
\begin{aligned}
\partial_y \cancel{E_z} - \partial_z E_y &= i \omega \mu_0 \mu_r H_x \\
\partial_z E_x - \partial_x E_z &= i \omega \mu_0 \mu_r H_y \\
\partial_x E_y - \partial_y \cancel{E_x} &= i \omega \mu_0 \mu_r H_z
\end{aligned}
\right.
\]

\[
\left\{
\begin{aligned}
\partial_y \cancel{H_z} - \partial_z H_y &= -i \omega \varepsilon_0 \varepsilon_r E_x \\
\partial_z H_x - \partial_x H_z &= -i \omega \varepsilon_0 \varepsilon_r E_y \\
\partial_x H_y - \partial_y \cancel{H_x} &= -i \omega \varepsilon_0 \varepsilon_r E_z
\end{aligned}
\right.
\]

On peut alors rassembler ce qui dépend de \(E_y\) d'un côté (transverse électrique) et de \(H_y\) de l'autre (transverse magnétique) : 
\\ \\
{Transverse électrique : TE} \hfill {Transverse magnétique : TM}

\[
\left\{
\begin{aligned}
-\partial_z E_y &= i\omega \mu_0 \mu_r H_x \\
\partial_z H_x - \partial_x H_z &= -i\omega \varepsilon_0 \varepsilon_r E_y \\
\partial_x E_y &= i\omega \mu_0 \mu_r H_z
\end{aligned}
\right.
\hspace{3cm}
\left\{
\begin{aligned}
-\partial_z H_y &= -i\omega \varepsilon_0 \varepsilon_r E_x \\
\partial_z E_x - \partial_x E_z &= i\omega \mu_0 \mu_r H_y \\
\partial_x H_y &= -i\omega \varepsilon_0 \varepsilon_r E_z
\end{aligned}
\right.
\]

\section*{Relation de dispersion}

On part de l'équation d’Alembert en deux dimensions (selon \(x\) et \(z\)) :

\[
\frac{\partial^2 u}{\partial x^2} + \frac{\partial^2 u}{\partial z^2} = \frac{1}{c^2} \frac{\partial^2 u}{\partial t^2}
\]

L'hypothèse d'une solution harmonique permet de traiter le membre de droite. Supposons une solution de la forme :

\[
u(x, z, t) = u(x, z)  e^{-i \omega t}.
\]

La dérivée seconde par rapport au temps devient alors :

\[
\frac{\partial^2 u}{\partial t^2} = -\omega^2 u(x, z)  e^{-i \omega t}
\]

En remplaçant dans l’équation d’Alembert, on obtient :

\[
\frac{\partial^2 u}{\partial x^2} + \frac{\partial^2 u}{\partial z^2} = -\frac{\omega^2}{c^2} u(x, z)
\]

Si le milieu est homogène, caractérisé par des constantes relatives \( \varepsilon_r \) et \( \mu_r \), on a \( c^2 = \frac{1}{\varepsilon_0 \mu_0} \), d’où :

\[
\frac{\omega^2}{c^2} \varepsilon_r \mu_r = \omega^2 \varepsilon_0 \mu_0 \varepsilon_r \mu_r
\]

Ainsi, l'équation devient :

\[
\frac{\partial^2 u}{\partial x^2} + \frac{\partial^2 u}{\partial z^2} = -\omega^2 \varepsilon_0 \mu_0 \varepsilon_r \mu_r  u(x, z)
\]

Les solutions sont des ondes planes de la forme :

\[
u(x, z) = u_0 e^{i \vec{k}  \vec{r}} = u_0 e^{i(k_x x + k_y y + k_z z)} = u_0 e^{i(\alpha x + \beta y + \gamma z)},
\]

avec les identifications :

\[
k_x = \alpha, \quad k_y = \beta, \quad k_z = \gamma
\]

On obtient alors la relation de dispersion :

\[
\alpha^2 + \beta^2 + \gamma^2 = \frac{\omega^2}{c^2} \varepsilon_r \mu_r
\]

Le théorème de Bloch affirme que la fonction d'onde est périodique à une phase près. Mathématiquement, cela s’écrit :

\[
u(x + d, z) = u(x, z) e^{i \alpha d}
\]

\section*{Système de coordonné en incidence conique :}

D'un coté le réseau avec lequel nous travaillons à une symétrie cartésienne, mais de l'autre nous souhaitons travailler en incidence conique ce qui necessite d'avoir des angles. Nous allons donc projetter la base conique dans la base catésienne pour pouvoir concilier ces deux aspects : 

\begin{align*}
x &= \cos\delta \cos\theta \cos\varphi - \sin\delta \sin\varphi \\
y &= \cos\delta \cos\theta \sin\varphi + \sin\delta \cos\varphi \\
z &= -\cos\delta \sin\theta
\end{align*}
\noindent
Les angles utilisés dans cette base conique sont définis comme suit :
\begin{itemize}
  \item \( \theta \) : angle polaire, mesuré par rapport à l’axe \( z \),
  \item \( \varphi \) : angle azimutal, mesuré dans le plan \( xOy \) : c'est cette angle qui nous fait sortir de l'incidence normale.
  \item \( \delta \) : angle secondaire : c'est cette angle qui nous fait sortir de l'incidence oblique pour nous placer en incidence conique. C'est cet angle qui defini dans quelle polarisation nous sommes.
\end{itemize}

Les composantes y du vecteur d'onde k ne sont plus nulles car on est en incidence conique.
\section*{Conditions aux limites à l'interface $z = 0$}

Les relations de passage indiquent que pour un matériau comme le graphène, qui ne possède pas de charge surfacique, il y a conservation de la composante tangentielle du champ $\mathbf{E}$ à l'interface ($z = 0$) :
\[
E_{1x}(x, y, 0) = E_{2x}(x, y, 0)
\]
\[
E_{1y}(x, y, 0) = E_{2y}(x, y, 0)
\]

En revanche, le graphène est conducteur. Les composantes tangentielles de $\mathbf{H}$ présentent donc un saut : elles obéissent en $z = 0$ à la loi d'Ohm locale :
\[
H_{2x}(x, y, 0) - H_{1x}(x, y, 0) = \sigma(x) E_{1/2\, y}(x, y, 0)
\]
\[
H_{2y}(x, y, 0) - H_{1y}(x, y, 0) = -\sigma(x)  E_{1/2\, x}(x, y, 0)
\]

On peut choisir librement $\mathbf{E}_1$ ou $\mathbf{E}_2$, car $\mathbf{E}_1 = \mathbf{E}_2$. D'où la notation $\mathbf{E}_{1/2}$.

\section*{Série de Fourier}

Le développement en série de Fourier impose :
\[
u(x, z) = \sum_{n \in \mathbb{Z}} u_n(z) e^{i \alpha_n x}
\]

Dans notre problème, on a deux fonctions d'onde :
\[
u_0(x, z) = \sum_{n \in \mathbb{Z}} \left( T_n e^{i \gamma_{on} z} + J_n e^{-i \gamma_{on} z} \right) e^{i \alpha_n x}
\]
et :

\[
u_i(x, z) = \sum_{n \in \mathbb{Z}} \left( I_n e^{i \gamma_{in} z} + R_n e^{-i \gamma_{in} z} \right) e^{i \alpha_n x}
\]

Les conventions de signe adoptées sont les suivantes :
\begin{itemize}
  \item Les ondes de la forme \( e^{+i \gamma z} \) correspondent aux ondes entrantes, c’est-à-dire celles qui se propagent vers l’intérieur de la structure.
  \item Les ondes de la forme \( e^{-i \gamma z} \) correspondent aux ondes sortantes, c’est-à-dire celles qui se propagent vers l’extérieur.
  \item \( u_i(x,z) \) désigne le champ dans le milieu 1, tandis que \( u_0(x,z) \) correspond au milieu 2.
\end{itemize}

\medskip


Dans ces expressions :
\begin{itemize}
  \item \( I_n \) et \( J_n \) sont les amplitudes des ondes \textbf{incidentes} dans les régions d'entrée et de sortie respectivement.
  \item \( R_n \) et \( T_n \) sont les amplitudes des ondes \textbf{réfléchies} et \textbf{transmises}.
\end{itemize}

\medskip

L’un des objectifs de ce projet est de déterminer les coefficients de réflexion \( R_n \) et de transmission \( T_n \)

Pour obtenir une relation entre ces coefficients, on se place à l'interface ($z = 0$) :
\[
\sum_{n \in \mathbb{Z}} \left( I_n + R_n \right) e^{i \alpha_n x}
=
\sum_{n \in \mathbb{Z}} \left( T_n + J_n \right) e^{i \alpha_n x}
\]

Ce qui donne :
\[
\forall n \in \mathbb{Z}, \quad I_n + R_n = T_n + J_n
\]
En effet, deux série de Fourier sont égales si et seulement si elles ont les mêmes coefficiants (même argument que dans l'algèbre des polynômes).
\section*{Expression de l’onde plane incidente}

L’onde plane incidente a pour expression :
\[
\mathbf{E}_{1+}(\mathbf{r}) = \mathbf{I}_0 e^{i \mathbf{k}_{10}^{+} \cdot \mathbf{r}}, \qquad
\mathbf{E}_{1-}(\mathbf{r}) = \mathbf{R}_n e^{i \mathbf{k}_{1n}^{-} \cdot \mathbf{r}}
\]
\[
\mathbf{E}_{2+}(\mathbf{r}) = \mathbf{T}_n e^{i \mathbf{k}_{2n}^{+} \cdot \mathbf{r}}, \qquad
\mathbf{E}_{2-}(\mathbf{r}) = \mathbf{J}_n e^{i \mathbf{k}_{2n}^{-} \cdot \mathbf{r}}
\]

On a donc :
\[
\mathbf{E}_1(\mathbf{r}) = \sum_{n \in \mathbb{Z}} \left( \mathbf{I}_n e^{i \mathbf{k}_{1n}^{+} \cdot \mathbf{r}} + \mathbf{R}_n e^{i \mathbf{k}_{1n}^{-} \cdot \mathbf{r}} \right)
\]
\[
\mathbf{E}_2(\mathbf{r}) = \sum_{n \in \mathbb{Z}} \left( \mathbf{T}_n e^{i \mathbf{k}_{2n}^{+} \cdot \mathbf{r}} + \mathbf{J}_n e^{i \mathbf{k}_{2n}^{-} \cdot \mathbf{r}} \right)
\]

Dans une onde électromagnétique plane, on a la relation suivante entre $\mathbf{k}$, $\mathbf{E}$, et $\mathbf{H}$ :
\[
\mathbf{H} = \frac{1}{\omega \mu_0} \mathbf{k} \wedge \mathbf{E}
\]

\medskip

Pour l’onde incidente :
\[
\mathbf{k}_{\text{inc}} = 
\begin{pmatrix}
k_1 \sin\theta \cos\phi \\
k_1 \sin\theta \sin\phi \\
k_1 \cos\theta
\end{pmatrix}
=
\begin{pmatrix}
\alpha \\
\beta \\
\gamma
\end{pmatrix}
\]

En toute generalité : 
\medskip
\[
\mathbf{k}_{pn}^{\pm} =
\begin{pmatrix}
\alpha_n \\
\beta \\
\pm \gamma_{pn}
\end{pmatrix}
=
\begin{pmatrix}
\alpha + n \dfrac{2\pi}{d} \\
\beta \\
\pm \sqrt{\left(k_0 \sqrt{\varepsilon_p \mu_p}\right)^2 - \alpha_n^2 - \beta^2}
\end{pmatrix}
\]

où $p$ est l'indice du milieu (1 ou 2). On constate que $\beta \neq 0$, contrairement au cas d'incidence normale.


On a donc : 

\[
\mathbf{H}_1(\mathbf{r}) = \frac{1}{\omega \mu_0} \sum_{n \in \mathbb{Z}} 
\left( \mathbf{k}_{1n}^{+} \wedge \mathbf{I}_n \, e^{i \mathbf{k}_{1n}^{+} \cdot \mathbf{r}} + 
       \mathbf{k}_{1n}^{-} \wedge \mathbf{R}_n \, e^{i \mathbf{k}_{1n}^{-} \cdot \mathbf{r}} \right)
\]

\[
\mathbf{H}_2(\mathbf{r}) = \frac{1}{\omega \mu_0} \sum_{n \in \mathbb{Z}} 
\left( \mathbf{k}_{2n}^{+} \wedge \mathbf{T}_n \, e^{i \mathbf{k}_{2n}^{+} \cdot \mathbf{r}} + 
       \mathbf{k}_{2n}^{-} \wedge \mathbf{J}_n \, e^{i \mathbf{k}_{2n}^{-} \cdot \mathbf{r}} \right)
\]

On a donc, pour la première onde (n = 1) :

\[
\mathbf{k}_{1,1}^{+} \wedge \mathbf{I}_1 =
\begin{pmatrix}
\beta I_{1,z} - \gamma_1 I_{1,y} \\
\gamma_1 I_{1,x} - \alpha I_{1,z} \\
\alpha I_{1,y} - \beta I_{1,x}
\end{pmatrix}
\]

Généralisé à un indice arbitraire \( n \), cela donne :

\[
\mathbf{k}_{1n}^{+} \wedge \mathbf{I}_n =
\begin{pmatrix}
\beta I_{z} - \gamma_1 I_{y} \\
\gamma_1 I_{x} - \alpha I_{z} \\
\alpha I_{y} - \beta I_{x}
\end{pmatrix}_n
\]

Le \( n \) indique l'indice de l'onde étudiée.

De la même manière, on a les relations :

\[
\mathbf{k}_{1n}^{-} \wedge \mathbf{R}_n =
\begin{pmatrix}
\beta R_{z} + \gamma_1 R_{y} \\
- \gamma_1 R_{x} - \alpha R_{z} \\
\alpha R_{y} - \beta R_{x}
\end{pmatrix}_n
\]

\[
\mathbf{k}_{2n}^{+} \wedge \mathbf{T}_n =
\begin{pmatrix}
\beta T_{z} - \gamma_2 T_{y} \\
\gamma_2 T_{x} - \alpha T_{z} \\
\alpha T_{y} - \beta T_{x}
\end{pmatrix}_n
\]

\[
\mathbf{k}_{2n}^{-} \wedge \mathbf{J}_n =
\begin{pmatrix}
\beta J_{z} + \gamma_2 J_{y} \\
- \gamma_2 J_{x} - \alpha J_{z} \\
\alpha J_{y} - \beta J_{x}
\end{pmatrix}_n
\]

Or, on a :

\[
k_0 = \frac{\omega}{c} = \frac{2\pi}{\lambda}
\]
\[
Z_0 = \sqrt{\frac{\mu_0}{\varepsilon_0}} 
\]

Ce qui donne: 
\[
\omega \mu_0 = k_0 Z_0
\]

On fait donc passer ce terme du côté gauche en divisant par \( k_0 Z_0 \) :

\[
\mathbf{H}_1(\mathbf{r}) = \frac{1}{k_0 Z_0} \sum_{n \in \mathbb{Z}} 
\left(
\begin{pmatrix}
\beta I_{z} - \gamma_1 I_{y} \\
\gamma_1 I_{x} - \alpha I_{z} \\
\alpha I_{y} - \beta I_{x}
\end{pmatrix}_n
e^{i \mathbf{k}_{1n}^{+} \cdot \mathbf{r}} +
\begin{pmatrix}
\beta R_{z} + \gamma_1 R_{y} \\
- \gamma_1 R_{x} - \alpha R_{z} \\
\alpha R_{y} - \beta R_{x}
\end{pmatrix}_n
e^{i \mathbf{k}_{1n}^{-} \cdot \mathbf{r}}
\right)
\]

\[
\mathbf{H}_2(\mathbf{r}) = \frac{1}{k_0 Z_0} \sum_{n \in \mathbb{Z}} 
\left(
\begin{pmatrix}
\beta T_{z} - \gamma_2 T_{y} \\
\gamma_2 T_{x} - \alpha T_{z} \\
\alpha T_{y} - \beta T_{x}
\end{pmatrix}_n
e^{i \mathbf{k}_{2n}^{+} \cdot \mathbf{r}} +
\begin{pmatrix}
\beta J_{z} + \gamma_2 J_{y} \\
- \gamma_2 J_{x} - \alpha J_{z} \\
\alpha J_{y} - \beta J_{x}
\end{pmatrix}_n
e^{i \mathbf{k}_{2n}^{-} \cdot \mathbf{r}}
\right)
\]



\section*{Expression de la conductivité \( \sigma_g \) :}

On considère la conductivité du graphène donnée par la somme des conductivités interbande et intrabande :

\[
\sigma_g = \sigma_{\text{intra}} + \sigma_{\text{inter}}
\]

\[
\sigma_{\text{intra}}(\omega) = \frac{2i e^2 k_B T}{\pi \hbar^2 (\omega + i \gamma)} \ln \left[ 2 \cosh\left(\frac{\mu_c}{2k_B T}\right) \right]
\]

\[
\sigma_{\text{inter}} = \frac{e^2}{4\hbar} \left( \frac{1}{2} + \frac{1}{\pi} \arctan \left( \frac{\hbar(\omega + i\gamma) - 2\mu_c}{2k_B T} \right)
- \frac{i}{2\pi} \ln \left( \frac{[\hbar(\omega + i\gamma) + 2\mu_c]^2}{[\hbar(\omega + i\gamma) - 2\mu_c]^2 + (2k_B T)^2} \right) \right)
\]

\bigskip

On a une conductivité \( \sigma_g(x) \) qui est \( d \)-périodique.  
L'idée est donc d'utiliser une matrice réduite de Toeplitz remplie avec les coefficients de Fourier de cette conductivité. 
\\ \\
Cette matrice se nomme aussi matrice de convolution (une convolution intervient losrsqu'on a le produit de deux serie de Fourier de même période d ; ce qui est le cas dans notre étude).
\\ \\
On a tronqué cette Matrice à M=4 dans la formule ci-dessous :

\[
\Lambda = 
\begin{pmatrix}
\sigma_0 & \sigma_{-1} & \sigma_{-2} & \sigma_{-3} & \sigma_{-4} \\
\sigma_1 & \sigma_0     & \sigma_{-1} & \sigma_{-2} & \sigma_{-3} \\
\sigma_2 & \sigma_1     & \sigma_0     & \sigma_{-1} & \sigma_{-2} \\
\sigma_3 & \sigma_2     & \sigma_1     & \sigma_0     & \sigma_{-1} \\
\sigma_4 & \sigma_3     & \sigma_2     & \sigma_1     & \sigma_0
\end{pmatrix}
\]


Voici comment calculer ces coefficients : \\ \\
Pour p=0 :
\[
\sigma_{n-p} = \sigma_0 \quad \text{(car sur la diagonale, \( n = p \)).}
\]

\[
\sigma_0 = \frac{1}{d} \int_0^d \sigma_g(x)\, dx = \left( \frac{a}{d} \right) \sigma_g
\]

Pour \( p \neq 0 \) :

\[
\sigma_p = \frac{1}{d} \int_0^d \sigma_g(x) e^{-iKp x} dx 
= \frac{\sigma_g}{d} \int_0^a e^{-iKp x} dx
\]

\[
= \frac{\sigma_g}{d} \left[ \frac{e^{-iKp x}}{-iKp} \right]_0^a 
= \frac{i \sigma_g}{d Kp} \left( e^{-iKp a} - 1 \right) 
= \frac{i \sigma_g}{2\pi p} \left( e^{-iKp a} - 1 \right)
\]

\bigskip

\[
\sigma_g(x) = 
\begin{cases}
\sum\limits_{p \in \mathbb{Z}} \sigma_p e^{iKp x} & \text{si } 0 \leq x \leq a \\
0 & \text{si } a \leq x \leq d
\end{cases}
\]

avec \( \sigma_p \) le coefficient de la série de Fourier :

\[
\sigma_p = \frac{1}{d} \int_0^d \sigma_g(x) e^{-iKp x} \, dx,
\quad \text{et} \quad K = \frac{2\pi}{d}
\]

\bigskip

\section*{Calcul des sorties du problème :}

Reprenons nos conditions aux limites : 

\[
\left\{
\begin{aligned}
E_{1x}(x, y, 0) &= E_{2x}(x, y, 0) \\
E_{1y}(x, y, 0) &= E_{2y}(x, y, 0)
\end{aligned}
\right.
\]

De ces relations on sort : 

\[
I_{xn} + R_{xn} = T_{xn} + J_{xn}
\]

\[
I_{yn} + R_{yn} = T_{yn} + J_{yn}
\]

Soit matriciellement : 

\[
\mathbf{I} + \mathbf{R} = \mathbf{T} + \mathbf{J}
\]


Et de plus : 
\[
\left\{
\begin{aligned}
H_{2x}(x, y, 0) - H_{1x}(x, y, 0) &= \sigma(x) \cdot E_{1/2\,y}(x, y, 0) \\
H_{2y}(x, y, 0) - H_{1y}(x, y, 0) &= -\sigma(x) \cdot E_{1/2\,x}(x, y, 0)
\end{aligned}
\right.
\]




De ces relations on obtient : 

\[
\left\{ 
(\beta T_z - \gamma_2 T_y) + (\beta J_z + \gamma_2 J_y) 
\right\}
- 
\left\{ 
(\beta I_z - \gamma_1 I_y) + (\beta R_z + \gamma_1 R_y) 
\right\}
= 
k_0 Z_0 \Lambda (T_y + J_y)
=
k_0 Z_0 \Lambda (I_y + R_y)
\]

\[
\left\{ 
(-\alpha T_z + \gamma_2 T_x) + (-\alpha J_z - \gamma_2 J_x) 
\right\}
- 
\left\{ 
(-\alpha I_z + \gamma_1 I_x) + (-\alpha R_z - \gamma_1 R_x) 
\right\}
= 
- k_0 Z_0 \Lambda (T_x + J_x)
=
k_0 Z_0 \Lambda (I_x + R_x)
\]
\\ \\
Remarque : Ces deux égalités sont essentielles car elles expriment les entrées du problème en fonction des inconnus T et R.

\bigskip

\section*{Expression des composantes du champ selon \( z \) :}

Le champ E est une onde plane d'expression : 


\[
\vec{E} = \vec{E}_0 e^{i(\vec{k} \cdot \vec{r} - \omega t)}
\]

Pour un milieu sans charge surfacique :

\[
\nabla \cdot \mathbf{E} = 0
\]
Ainsi : 
\[
\nabla \cdot \vec{E} = \nabla \cdot \left( \vec{E}_0 e^{i(\vec{k} \cdot \vec{r} - \omega t)} \right)
= i \vec{E}_0 \cdot \vec{k} e^{i(\vec{k} \cdot \vec{r} - \omega t)}
\]

On prend la partie réelle : 
\[
\nabla \cdot \vec{E} = i \vec{E}_0 \cdot \vec{k} e^{i(\vec{k} \cdot \vec{r} - \omega t)}
\]
Ce qui donne : 
\[
\nabla \cdot \vec{E} = \vec{k} \cdot \vec{E}=0
\]

D'où :

\[
E_z = -\frac{1}{k_z} \left( k_x E_x + k_y E_y \right)
\]

\bigskip

On a donc ces équations que l'on peut injecter dans les conditions de passage :

\[
I_z = -\frac{\alpha I_x + \beta I_y}{\gamma_1}, \quad
J_z = \frac{\alpha J_x + \beta J_y}{\gamma_2}, \quad
T_z = -\frac{\alpha T_x + \beta T_y}{\gamma_2}, \quad
R_z = \frac{\alpha R_x + \beta R_y}{\gamma_1}
\]

Ce qui donne : 

\[
\left\{
\begin{array}{l}
\beta T_z - \gamma_2 T_y = \beta \left( \frac{-\alpha T_x - \beta T_y}{\gamma_2} \right) - \gamma_2 T_y 
= -\frac{\alpha \beta}{\gamma_2} T_x - \left( \gamma_2 + \frac{\beta^2}{\gamma_2} \right) T_y \\[1ex]

\beta J_z + \gamma_2 J_y = \beta \left( \frac{\alpha J_x + \beta J_y}{\gamma_2} \right) + \gamma_2 J_y 
= \frac{\alpha \beta}{\gamma_2} J_x + \left( \gamma_2 + \frac{\beta^2}{\gamma_2} \right) J_y \\[1ex]

-\alpha T_z + \gamma_2 T_x = -\alpha \left( \frac{-\alpha T_x - \beta T_y}{\gamma_2} \right) + \gamma_2 T_x 
= \left( \gamma_2 + \frac{\alpha^2}{\gamma_2} \right) T_x + \frac{\alpha \beta}{\gamma_2} T_y \\[1ex]

-\alpha J_z - \gamma_2 J_x = -\alpha \left( \frac{\alpha J_x + \beta J_y}{\gamma_2} \right) - \gamma_2 J_x 
= -\left( \gamma_2 + \frac{\alpha^2}{\gamma_2} \right) J_x - \frac{\alpha \beta}{\gamma_2} J_y \\[1ex]

\beta I_z - \gamma_1 I_y = \beta \left( \frac{-\alpha I_x + \beta I_y}{\gamma_1} \right) - \gamma_1 I_y 
= -\frac{\alpha \beta}{\gamma_1} I_x - \left( \gamma_1 + \frac{\beta^2}{\gamma_1} \right) I_y \\[1ex]

\beta R_z + \gamma_1 R_y = \beta \left( \frac{\alpha R_x + \beta R_y}{\gamma_1} \right) + \gamma_1 R_y 
= \frac{\alpha \beta}{\gamma_1} R_x + \left( \gamma_1 + \frac{\beta^2}{\gamma_1} \right) R_y \\[1ex]

-\alpha I_z + \gamma_1 I_x = -\alpha \left( \frac{-\alpha I_x + \beta I_y}{\gamma_1} \right) + \gamma_1 I_x 
= \left( \gamma_1 + \frac{\alpha^2}{\gamma_1} \right) I_x + \frac{\alpha \beta}{\gamma_1} I_y \\[1ex]

-\alpha R_z - \gamma_1 R_x = -\alpha \left( \frac{\alpha R_x + \beta R_y}{\gamma_1} \right) - \gamma_1 R_x 
= -\left( \gamma_1 + \frac{\alpha^2}{\gamma_1} \right) R_x - \frac{\alpha \beta}{\gamma_1} R_y
\end{array}
\right.
\]

Matriciellement cela donne  : 
\begin{flushleft}
\(
\[
-
\begin{pmatrix}
\frac{\alpha \beta}{\gamma_2} & \gamma_2 + \frac{\beta^2}{\gamma_2} \\
\gamma_2 + \frac{\alpha^2}{\gamma_2} & \frac{\alpha \beta}{\gamma_2}
\end{pmatrix}
\begin{pmatrix}
T_x \\ T_y
\end{pmatrix}
+
\begin{pmatrix}
\frac{\alpha \beta}{\gamma_2} & \gamma_2 + \frac{\beta^2}{\gamma_2} \\
\gamma_2 + \frac{\alpha^2}{\gamma_2} & \frac{\alpha \beta}{\gamma_2}
\end{pmatrix}
\begin{pmatrix}
J_x \\ J_y
\end{pmatrix}
+
\begin{pmatrix}
\frac{\alpha \beta}{\gamma_1} & \gamma_1 + \frac{\beta^2}{\gamma_1} \\
\gamma_1 + \frac{\alpha^2}{\gamma_1} & \frac{\alpha \beta}{\gamma_1}
\end{pmatrix}
\begin{pmatrix}
I_x \\ I_y
\end{pmatrix}
-
\begin{pmatrix}
\frac{\alpha \beta}{\gamma_1} & \gamma_1 + \frac{\beta^2}{\gamma_1} \\
\gamma_1 + \frac{\alpha^2}{\gamma_1} & \frac{\alpha \beta}{\gamma_1}
\end{pmatrix}
\begin{pmatrix}
R_x \\ R_y
\end{pmatrix}
=
\begin{pmatrix}
0 & k_0 Z_0 \Lambda \\
k_0 Z_0 \Lambda & 0
\end{pmatrix}
\begin{pmatrix}
T_x \\ T_y
\end{pmatrix}
+
\begin{pmatrix}
0 & k_0 Z_0 \Lambda \\
k_0 Z_0 \Lambda & 0
\end{pmatrix}
\begin{pmatrix}
J_x \\ J_y
\end{pmatrix}
\]
\end{flushleft}
\[

On identifie :

\[
\Gamma_1 =
\begin{pmatrix}
\frac{\alpha \beta}{\gamma_1} & \gamma_1 + \frac{\beta^2}{\gamma_1} \\
\gamma_1 + \frac{\alpha^2}{\gamma_1} & \frac{\alpha \beta}{\gamma_1}
\end{pmatrix}
\]

\[
\Gamma_2 =
\begin{pmatrix}
\frac{\alpha \beta}{\gamma_2} & \gamma_2 + \frac{\beta^2}{\gamma_2} \\
\gamma_2 + \frac{\alpha^2}{\gamma_2} & \frac{\alpha \beta}{\gamma_2}
\end{pmatrix}
\]

\[
\Omega =
\begin{pmatrix}
0 & k_0 Z_0 \Lambda \\
k_0 Z_0 \Lambda & 0
\end{pmatrix}
\]
\bigskip
\\
Ce qui donne l'égalité matriciel finale : 

\[
\Gamma_2(-T + J) + \Gamma_1(I - R) = \Omega (T + J)
\]
\section*{Résultat final :}
\bigskip
\[
\left\{
\begin{aligned}
-I + J & = R - T  \\
 \Gamma_1 I + (\Gamma_2 - \Omega) J & = \Gamma_1 R + (\Gamma_2 + \Omega) T
\end{aligned}
\right.
\]
\bigskip
Les membres à gauche correspondent aux entrées. Ceux de droite aux sorties.

Pour des raisons historiques on peut réarranger ce système d'équation matriciel sous forme d'une matrice "S" : 

\bigskip

S =
\left \mathbf{\begin{pmatrix}
\mathbf{I} & -\mathbf{I}  \\
\Gamma_1 & \Gamma_2 + \Omega
\end{pmatrix}} \right^{-1}
{\begin{pmatrix}
-\mathbf{I} & \mathbf{I}  \\
\Gamma_1 & \Gamma_2 - \Omega
\end{pmatrix}} \\ \\ 

On a finallement : \\

\[
\boxed{\left( \begin{matrix} R \\ T \end{matrix} \right)
= 
S \begin{pmatrix}
I \\
J
\end{pmatrix}}
\]




\section*{Absorption :}
\bigskip
Un milieu totalement absorbant a une absorption \( \mathcal{A} = 1 \) :

\[
\mathcal{A} = 1 - \sum_n \left( \mathcal{E}_{rn} + \mathcal{E}_{tn} \right)
\]

où : 

\[
\mathcal{E}_{rn} = \frac{\Phi_{rn}}{\Phi_i}, 
\qquad
\mathcal{E}_{tn} = \frac{\Phi_{tn}}{\Phi_i}
\]

La part de puissance transportée par l'onde diminue cette absorption.
\bigskip
\section*{Conclusion :}
\bigskip

\end{document}