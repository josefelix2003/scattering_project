\documentclass{article}
\usepackage{amsmath}
\usepackage{amssymb}

\begin{document}

Pour un mode Transverse Électrique (\textbf{TE}) :

\[

\begin{cases}
\vec{E} = E_y(x,z) \, \vec{e}_y \\
\vec{H} = H_x \, \vec{e}_x + H_z \, \vec{e}_z
\end{cases}
\]
\[
r_{TE} = \frac{\gamma_1 - \gamma_2}{\gamma_1 + \gamma_2}
\quad \text{et} \quad
t_{TE} = \frac{2\gamma_1}{\gamma_1 + \gamma_2}
\]
\[
\gamma_{1/2} = \sqrt{\frac{\omega^2}{c^2} \varepsilon_{1/2} - \alpha^2}
\]
\[
\alpha = \left( \frac{\omega}{c} n_1 \right) \sin \theta = (k_1)\sin \theta
\]

Donc : 
\[
\gamma_{1/2} = \sqrt{k_{1/2}^2 - \alpha^2}
\]
\[
k_{1/2} = \left( \frac{\omega}{c} \right) \, n_{1/2} = (k_0) \, n_{1/2}
\]
\[
\text{En incidence normale : } (\theta = 0) :
\]
\[
\text{Donc } \alpha = 0
\]
\[
\gamma_{1/2} = \sqrt{k_{1/2}^2 - 0^2} = k_{1/2}
\]
Le coefficient de réflexion en polarisation TE, sous incidence normale, est donné par :
\[
r_{TE} = \frac{k_1 - k_2}{k_1 + k_2}= 
\frac{\gamma_1 - \gamma_2}{\gamma_1 + \gamma_2}
= \frac{k_0 n_1 - k_0 n_2}{k_0 n_1 + k_0 n_2}
= \frac{n_1 - n_2}{n_1 + n_2}
\]

En l'absence de graphène, et pour \( n_1 = 1 \) et \( n_2 = 1{,}5 \) : \\ \\
\[
r_{TE} = \frac{1 - 1{,}5}{1 + 1{,}5}=-0,2
\]

\varepsilon_r =  \frac{n_1}{n_1}|r_{TE}|^2 
= |{-0{,}2}|^2 = 0{,}04 \equiv 4\% \\\

De la même manière : \\

t_{TE} = \frac{2 \gamma_1}{\gamma_1 + \gamma_2}
= \frac{2 k_1}{k_1 + k_2}
= \frac{2 k_0 n_1}{k_0 n_1 + k_0 n_2}
= \frac{2 n_1}{n_1 + n_2}
= 0{,}8 \\\\
\[
\varepsilon_t = \frac{n_2}{n_1} \left| t_{TE} \right|^2=\frac{1{,}5}{1} \times |0{,8}|^2=0.96=96\% \\\
\]

Donc : \\

\varepsilon_t + \varepsilon_r =1 \\

En incidence classique (\( \theta \ne 0 \)) :
\[
r_{TE} = \frac{\gamma_1 - \gamma_2}{\gamma_1 + \gamma_2}
= \frac{
\sqrt{k_1^2 - k_1^2 \sin^2 \theta}
-
\sqrt{k_2^2 - k_1^2 \sin^2 \theta}
}{
\sqrt{k_1^2 - k_1^2 \sin^2 \theta}
+
\sqrt{k_2^2 - k_1^2 \sin^2 \theta}
}
\]
\[
\frac{k_1}{k_2} = \frac{k_0 n_1}{k_0 n_2} = \frac{n_1}{n_2}
\]
Loi de Snell-Descartes : 

\[
n_1 \sin(\theta_1) = n_2 \sin(\theta_2)
\]

\[
\frac{n_1}{n_2} \sin(\theta_1) = \sin(\theta_2)
\]

\[
r_{TE} = \frac{
k_1 \sqrt{1 - \sin^2 \theta_1} - k_2 \sqrt{1 - \sin^2 \theta_2}
}{
k_1 \cos(\theta_1) + k_2 \cos(\theta_2)
}
\]
On obtient ainsi :
\[
r_{TE} = \frac{n_1 \cos \theta_1 - n_2 \cos \theta_2}{n_1 \cos \theta_1 + n_2 \cos \theta_2}
\]\\\

Pour un mode Transverse Magnetique (\textbf{TM}) :

\[

\left\{
\begin{array}{l}
\vec{H} = H_y(x,z)\, \vec{e}_y \\
\vec{E} = E_x\, \vec{e}_x + E_z\, \vec{e}_z
\end{array}
\right.
\]
\[
r_{TM} = \frac{\gamma_1' - \gamma_2'}{\gamma_1' + \gamma_2'}
\quad \text{et} \quad
t_{TM} = \frac{2 \gamma_1'}{\gamma_1' + \gamma_2'}
\]
\[
\gamma_1' = \frac{\gamma_1}{\varepsilon_1}=\frac{\gamma_1}{n_1^2}
\]
\[
\gamma_2' = \frac{\gamma_2}{\varepsilon_2}=\frac{\gamma_2}{n_2^2}
\]
En incidence conique, on permute :
\[
(\theta = 30^\circ, \psi = 0^\circ) \quad \leftrightarrow \quad (\theta = 30^\circ, \psi = 45^\circ)
\]

On effectue ce test avec :
\[
\delta = 90^\circ \quad \Rightarrow \quad \text{polarisation TE}
\]
\[
\delta = 0^\circ \quad \Rightarrow \quad \text{polarisation TM}
\]



\end{document}
